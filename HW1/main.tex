\documentclass[a4paper,11pt]{article}

\usepackage{listings}
\usepackage{lmodern}
\usepackage{amsmath,amssymb,amsthm,textcomp}
\usepackage[utf8]{inputenc}  
\usepackage[T1]{fontenc}  
\usepackage{ listings }
\usepackage{graphicx}
\usepackage{float}
\usepackage{enumitem}
\usepackage{stmaryrd}
\usepackage{bbm}
\graphicspath{ {images/} }

\usepackage{geometry}
\geometry{total={210mm,297mm},
textwidth=16cm,%
bindingoffset=0mm, top=20mm,bottom=20mm}

\usepackage{etoolbox}
\makeatletter
\preto{\@verbatim}{\topsep=0pt \partopsep=3pt }
\makeatother

\linespread{1}

\newcommand{\linia}{\rule{\linewidth}{0.5pt}}
\setlength{\parindent}{0pt}

% my own titles
\makeatletter
\renewcommand{\maketitle}{
\begin{center}
\vspace{2ex}
{\huge \textsc{\@title}}
\vspace{1ex}
\\
\linia\\
\@author 
\vspace{4ex}
\end{center}
}
\makeatother
%%%

% custom footers and headers
\usepackage{fancyhdr}
\pagestyle{fancy}
\lhead{}
\chead{}
\rhead{}

\cfoot{}
\rfoot{Page \thepage}
\renewcommand{\headrulewidth}{0pt}
\renewcommand{\footrulewidth}{0pt}
%
\DeclareMathOperator*{\argmax}{arg\,max}
\DeclareMathOperator*{\argmin}{arg\,min}
\let\o\relax
\DeclareMathOperator*{\o}{\mathit o_{\mathbb P}}
\let\O\relax
\DeclareMathOperator*{\O}{\mathit O_{\mathbb P}}
\DeclareMathOperator*{\Log}{Log}
%%%----------%%%----------%%%----------%%%----------%%%

\begin{document}

\title{HW1: Asymptotics}

\author{Gabriel ROMON}



\maketitle

\section*{Problem 1}

\noindent\fbox{%
    \parbox{\textwidth}{%
    Let $(T_n)_{n\geq 1}$ be a sequence of random vectors of $\mathbb R^d$ ($d\geq 1$). $T_n$ is said to be \textit{bounded in probability} or \textit{tight} (denoted $T_n=\O(1)$) if for any $\varepsilon>0$, there is some $A>0$ and $n_0\geq 1$ such that $n\geq n_0\implies P(\|T_n\|_2\geq A)\leq \varepsilon$. More generally, if $(s_n)_{n\geq 1}$ is a sequence of real random variables we write $T_n=\O(s_n)$ if for any $\varepsilon>0$, there is some $A>0$ and $n_0\geq 1$ such that $n\geq n_0\implies P(\|T_n\|_2\geq A s_n)\leq \varepsilon$.
    \begin{enumerate}
      \item Show that if $T_n=\o(1)$ then $T_n=\O(1)$.
      \item Show that if $T_n$ converges in probability, $T_n$ is tight.
      \item Show that if $T_n$ converges in distribution, $T_n$ is tight.
      \item Show that if $(\rho_n)_{n\geq 1}$ is a sequence that goes to $\infty$ and $\rho_n T_n$ converges in distribution, then $T_n=\o(1)$.
      \item Suppose that $T_n$ goes to $0$ in probability. Let $g:\mathbb R^d\to \mathbb R$ be such that $g(x)=o(\|x\|_2^p)$ as $x\to 0$. Show that $g(T_n)=\o(\|T_n\|_2^p)$.
      \item Suppose that $T_n$ goes to $0$ in probability. Let $g:\mathbb R^d\to \mathbb R$ be such that $g(x)=O(\|x\|_2^p)$ as $x\to 0$. Show that $g(T_n)=\O(\|T_n\|_2^p)$
      \item Let $(X_n)_{n\geq 1}$ be a sequence of r.v.'s such that $X_n\sim \mathcal P(\frac 1n)$.
      \begin{enumerate}
        \item Show that $X_n=\o(1)$.
        \item Show that for any sequence $(u_n)_{n\geq 1}$ of positive reals, $X_n=\o(u_n)$.
        \item Does $X_n$ converge almost surely ?
      \end{enumerate}
    \end{enumerate}
    }%
}

\begin{enumerate}
  \item Let $\varepsilon>0$. Since $T_n=\o(1)$, the sequence $P(\|T_n\|_2\geq \varepsilon)$ goes to $0$. There exists $n_0$ such that $n\geq n_0\implies P(\|T_n\|_2\geq \varepsilon) \leq \varepsilon$ and we're done.

  \item \underline{Lemma 1}: If $X_n=\O(1)$ and $Y_n=\O(1)$ then $X+Y=\O(1)$. \newline
  \textit{Proof}: Let $\varepsilon>0$. There are some $A_1, A_2>0$ and $n_1,n_2\geq 1$ such that $n\geq n_1 \implies P(\|X_n\|_2\geq A_1)\leq \frac \varepsilon 2$ and $n\geq n_2 \implies P(\|Y_n\|_2\geq A_2)\leq \frac \varepsilon 2$. Let $A=\max(A_1,A_2)$, $n_0=\max(n_1,n_2)$ and note that for $n\geq n_0$: 
  $$\begin{aligned}
  P(\|X_n+Y_n\|_2\geq 2A) &\leq P(\|X_n\|_2+\|Y_n\|_2\geq 2A)\\ 
  &\leq P(\|X_n\|_2\geq A) + P(\|Y_n\|_2\geq A) \\
  &\leq P(\|X_n\|_2\geq A_1) + P(\|Y_n\|_2\geq A_2)\\
  &\leq \varepsilon
  \end{aligned}
  $$
  \hfill $\square$\newline
  Let $T$ denote the limit of $T_n$ in probability. We have $T_n-T=\o(1)$, hence 1. yields $T_n-T=\O(1)$. Since $T_n = T_n-T+T$ and $T=\O(1)$, the lemma yields $T_n = \O(1)$. 

  \item \underline{Lemma 2}: Let $(X_n^1,\ldots,X_n^d)$ be a random vector. $(X_n^1,\ldots,X_n^d)$ is tight if and only if each of the $X_n^i$ is tight.\newline
  \textit{Proof}: $\Rightarrow$ Let $\varepsilon>0$. There are some $A>0$ and $n_0\geq 1$ such that $n\geq n_0\implies P(\|X_n\|_2\geq A)\leq \varepsilon$. For $n\geq n_0$, $P(|X_n^i|\geq A) \leq P(\|X_n\|_\infty \geq A) \leq P(\|X_n\|_2 \geq A) \leq \varepsilon$ and we're done.\newline
  $\Leftarrow$ Let $\varepsilon>0$. There are some $A_1,\ldots,A_d$ and $n_1,\ldots,n_d$ linked to the $\displaystyle \frac{\varepsilon}d$ tightness of each $X^i$. Let $A=\sqrt d \max_{1\leq i \leq d} A_i$ and $n_0 = \max_{1\leq i \leq d} n_i$. For $n\geq n_0$ we have 
  $$\begin{aligned}
    P(\|X_n\|_2\geq A) &\leq P(\sqrt d \|X_n\|_\infty \geq A) \\
    &= P(\bigcup_{i=1}^d |X_n^i| \geq \frac A{\sqrt d})\\
    &\leq \sum_{i=1}^d  P(|X_n^i| \geq \frac A{\sqrt d})\\
    &\leq \varepsilon
  \end{aligned}$$
  \hfill $\square$\newline
  Let us prove the result for $d=1$. By Lemma 1 it suffices to prove that if $T_n\in \mathbb R$ converges to $0$ in distribution, then $T_n$ is tight. The cdf of $0$ is continuous everywhere except at $0$, so $P(T_n\leq 1)$ goes to $1$ as $n\to \infty$. Consequently $P(T_n\geq 2)\to 0$, hence $T_n$ is tight.\\ \\
  Let $d\geq 2$ and $T_n$ be a sequence that converges in distribution. By the continous mapping theorem, each $T_n^i$ converges in distribution, hence each $T_n^i$ is tight. By Lemma 2, $T_n$ is tight.

  \item Let $\varepsilon >0$ and $X$ denote a random variable having the distribution of the limit of $\rho_n T_n$. By the continuous mapping theorem $\|\rho_n T_n\|_2$ converges in distribution to $\|X\|_2$. Since $\rho_n\to \infty$, we may assume WLOG that $\rho_n\geq 0$. Let $A>0$ be fixed. There exists some $n_0$ such that $n\geq n_0 \implies \rho_n \geq A$. For $n\geq n_0$, 
  $$\begin{aligned}
    P(\|T_n\|_2 \geq \varepsilon) &= P(\rho_n\|T_n\|_2 \geq \varepsilon\rho_n)\\
    &\leq P(\rho_n\|T_n\|_2 \geq \varepsilon A)
  \end{aligned}$$
  Taking the $\limsup$ on both side yields $\limsup_n P(\|T_n\|_2 \geq \varepsilon) \leq \limsup_n P(\rho_n\|T_n\|_2 \geq \varepsilon A)$.\newline The portmanteau theorem applied to $\rho_n\|T_n\|_2$ and the closed set $[\varepsilon A, \infty)$ gives $$\limsup_n P(\rho_n\|T_n\|_2 \geq \varepsilon A) \leq P(\|X\|_2 \geq \varepsilon A)$$
  hence $\limsup_n P(\|T_n\|_2 \geq \varepsilon) \leq P(\|X\|_2 \geq \varepsilon A)$.\newline Letting $A\to \infty$ yields $\limsup_n P(\|T_n\|_2 \geq \varepsilon) = 0$, hence $P(\|T_n\|_2 \geq \varepsilon)\to 0$ and $T_n=\o(1)$. 

  \item Let $\varepsilon >0$. Since $g(x)=o(\|x\|_2^p)$ as $x\to 0$, there exists some $R$ such that\newline $x\in B_2(0,R)\setminus\{0\}\implies \frac{|g(x)|}{\|x\|_2^p}\leq \frac \varepsilon 2$. This implies 
  $$
  P(g(T_n)\geq \varepsilon \|T_n\|_2^p) \leq P(\|T_n\|_2> R) \xrightarrow[n\to \infty]{} 0
  $$
  and we're done.

  \item Let $\varepsilon >0$. Since $g(x)=O(\|x\|_2^p)$ as $x\to 0$, there exists some $A>0$ and $R>0$ such that\newline $x\in B_2(0,R)\setminus\{0\}\implies \frac{|g(x)|}{\|x\|_2^p}\leq A$. This implies 
  $$
  P(g(T_n)\geq 2A \|T_n\|_2^p) \leq P(\|T_n\|_2> R) \xrightarrow[n\to \infty]{} 0
  $$
  Consequently there is some $n_0$ such that $n\geq n_0 \implies P(\|T_n\|_2> R) \leq \varepsilon$. Hence $n\geq n_0 \implies P(g(T_n)\geq 2A \|T_n\|_2^p) \leq \varepsilon$ and we're done.

  \item 
  \begin{enumerate}
    \item Let $\varepsilon \in (0,1)$. Since $X_n$ is integer-valued, we have $$P(X_n\geq \varepsilon) \leq P(X_n\geq 1) = 1-P(X_n=0) = 1-\exp(-\frac 1n) \xrightarrow[n\to \infty]{} 0$$
    For $\varepsilon \geq 1$, $P(X_n\geq \varepsilon) \leq P(X_n\geq \frac 12) \xrightarrow[n\to \infty]{} 0$, hence $X_n$ converges in probability to $0$.
    \item Let $(u_n)_{n\geq 1}$ be a sequence of positive reals and $\varepsilon >0$. Note that 
    $$\begin{aligned}
      P(X_n\geq \varepsilon u_n) &= P(X_n\geq \varepsilon u_n) 1_{u_n>\frac{1}{\varepsilon}} + \sum_{k=0}^\infty P(X_n\geq \varepsilon u_n) 1_{\frac{1}{2^k \varepsilon} \geq u_n>\frac{1}{2^{k+1} \varepsilon}}\\
      &\leq P(X_n\geq 1) + \sum_{k=0}^\infty P\left(X_n\geq \frac{1}{2^{k+1}}\right) 1_{\frac{1}{2^k \varepsilon} \geq u_n>\frac{1}{2^{k+1} \varepsilon}}\\
      &\leq P(X_n\geq 1) + \sum_{k=0}^\infty \left(1-\exp(-\frac 1n)\right) 1_{\frac{1}{2^k \varepsilon} \geq u_n>\frac{1}{2^{k+1} \varepsilon}}\\
      &\leq P(X_n\geq 1) + \sum_{k=0}^\infty \frac 1n \; 1_{\frac{1}{2^k \varepsilon} \geq u_n>\frac{1}{2^{k+1} \varepsilon}}\\
      &\leq P(X_n\geq 1) + \frac 1n\\
      &\xrightarrow[n\to \infty]{} 0
    \end{aligned}$$
    Hence $X_n=\o(u_n)$.

    \item Note that $P(X_n\geq 1) = 1-\exp(-\frac 1n)\sim \frac 1n$, hence $\sum_n P(X_n\geq 1) = \infty$.\\
    \textbf{If} the $X_i$ are (at least pairwise) \textbf{independent}, Borel-Cantelli lemma yields \\
    $P\left(\limsup_n \left(X_n\geq 1\right)\right)=1$. For almost all $w$, we have $X_n(w)\geq 1$ infinitely often, hence $X_n$ cannot converge almost surely to $0$. Thus $X_n$ does not converge almost surely.\\\\
    If the $X_i$ are not independent, a.s. convergence to $0$ may occur. Let $(\xi_n)_{n\geq 1}$ be a sequence of independent r.v's such that $\xi_n \sim \mathcal P(\frac{1}{n}-\frac{1}{n+1})$. By Kolmogorov's three series theorem, the series $\sum_{n\geq 1} \xi_n$ converges almost surely. Indeed 
    $$\begin{aligned}P(\xi_n \geq 1) &= 1 - \exp(\frac{1}{n}-\frac{1}{n+1}) \sim \frac{1}{n^2}\\
    E(X_n 1_{X_n\leq 1}) &= E(1_{X_n= 1}) = P(X_n= 1)=\left(\frac{1}{n}-\frac{1}{n+1}\right)\exp(\frac{1}{n}-\frac{1}{n+1}) \sim \frac{1}{n^2}\\
    V(X_n 1_{X_n\leq 1}) &= E(X_n 1_{X_n\leq 1}) - E(X_n 1_{X_n\leq 1})^2 \sim \frac{1}{n^2}
    \end{aligned}
    $$
    Thus it makes sense to define $X_n=\sum_{i\geq n} \xi_i$, and by what precedes $X_n$ converges almost surely to $0$.\\
    It remains to prove that $X_n\sim \mathcal P(\frac 1n)$. 
    $$\begin{aligned}P(X_n=p) &= E(1_{X_n=p}) =  E(\lim_k1_{\sum_{i=n}^k\xi_i=p})\\
&= \lim_k P(\sum_{i=n}^k\xi_i=p)\\
&= \lim_k \frac{1}{p!}\left(\frac 1n - \frac 1{k+1}\right)^p \exp\left(-(\frac 1n - \frac 1{k+1})\right) \\
&= \frac{1}{p!}\frac 1{n^p} \exp(-\frac 1n)
\end{aligned}$$
    Swapping the $\lim_k$ and the expectation is motivated by the monotone convergence theorem. For fixed $k$, the distribution of $\sum_{i=n}^k\xi_i$ is that of a finite sum of independent Poisson random variables, which is well-known to be a Poisson where the parameters are summed.
    \end{enumerate}
  
\end{enumerate}

\section*{Problem 2}
\noindent\fbox{%
    \parbox{\textwidth}{%
    Let $(T_n)_{n\geq 1}$ be a sequence of random vectors of $\mathbb R^d$ ($d\geq 1$) and $T$ a random vector.
    \begin{enumerate}
      \item Show that if $T_n$ converges almost surely to $T$, then $T_n$ converges in probability to $T$.
      \item Show that if $T_n$ converges in probability to $T$, then $T_n$ converges in distribution to $T$.
      \item Show that if $T$ is constant almost surely, convergence in distribution implies convergence in probability.
    \end{enumerate}
    }%
}

\begin{enumerate}
  \item Let $A = \{w\in \Omega,\; T_n(w)\xrightarrow[n\to \infty]{} T(w)\}$. Note that $A = \bigcap_{m\geq 1} \bigcup_{n\geq 1} \bigcap_{k\geq n} \|T_k-T\|< \frac 1m$.\newline By assumption, $P(A^c)=0$, hence $P(\bigcup_{m\geq 1} \bigcap_{n\geq 1} \bigcup_{k\geq n} \|T_k-T\|\geq \frac 1m) = 0$, which implies $\forall m\geq 1, P(\bigcap_{n\geq 1} \bigcup_{k\geq n} \|T_k-T\|\geq \frac 1m) = 0$.

  Let $\varepsilon >0$. There is some $m\geq 1$ such that $\varepsilon \geq \frac 1m$, hence 
  $$P(\|T_n-T\|\geq \varepsilon) \leq P(\|T_n-T\|\geq \frac 1m) \leq P(\bigcup_{k\geq n} \|T_k-T\|\geq \frac 1m)$$
  $\bigcup_{k\geq n} \|T_k-T\|\geq \frac 1m$ is a decreasing sequence of events, hence $$P(\bigcup_{k\geq n} \|T_k-T\|\geq \frac 1m) \xrightarrow[n\to \infty]{} P(\bigcap_{n\geq 1} \bigcup_{k\geq n} \|T_k-T\|\geq \frac 1m) = 0$$
  Squeezing thus yields $P(\|T_n-T\|\geq \varepsilon) \xrightarrow[n\to \infty]{} 0$, hence $T_n$ converges to $T$ in probability.

  \item Let $f$ be a $K$-Lipschitz function bounded by some $A\geq 0$. Note that for any $\varepsilon >0$, $$|f(T_n)-f(T)|\leq K\varepsilon 1_{\|T_n-T\|_2\leq \varepsilon} + 2A \varepsilon 1_{\|T_n-T\|_2> \varepsilon}$$
  Thus $|E(f(T_n)) - E(f(T))|\leq E(|f(T_n)-f(T)|)\leq K\varepsilon + 2A \varepsilon P(\|T_n-T\|_2> \varepsilon)$.\newline
  Taking $\limsup$ on both side yields $\limsup_n |E(f(T_n)) - E(f(T))|\leq K\varepsilon $. Letting $\varepsilon \to 0$ proves that $\lim_n E(f(T_n)) = E(f(T))$. By the portmanteau theorem, $T_n$ converges to $T$ in distribution.

  \item If the random vector $(T_n^1,\ldots, T_n^d)$ converges in distribution to some $T$ with $T=(t_1,\ldots,t_d)$ a.s., then by the continuous mapping theorem each $T_n^i$ converges in distribution to $\delta_{t_i}$.

  We recall a useful lemma about convergence:\newline
  \underline{Lemma 3}: $(X_n^1,\ldots,X_n^d)$ converges in probability to $(X^1,\ldots,X^d)$ if and only if each real r.v. $X_n^i$ converges in probability to $X^i$. \newline
  By Lemma $3$ it suffices to prove the claim in the case $d=1$.\\
  By the continuous mapping theorem applied with $x\mapsto |x-t|$, $|T_n-t|$ converges in distribution to $0$. Let $\varepsilon >0$ and note that $$P(|T_n-t|\geq \varepsilon) \leq P(|T_n-t|> \frac{\varepsilon}2) = 1-P(|T_n-t|\leq \frac{\varepsilon}2)$$
  Since the cdf of $0$ is continuous at $\displaystyle \frac{\varepsilon}2$, the convergence of $|T_n-t|$ implies $\displaystyle P(|T_n-t|\leq \frac{\varepsilon}2) \xrightarrow[n\to \infty]{} 1$ and we're done.
\end{enumerate}



\section*{Problem 3}
\noindent\fbox{%
    \parbox{\textwidth}{%
    Let $\alpha\in (0,1)$, $\ell_\alpha:t\mapsto (1-\alpha)t^++\alpha t^-$ and $\phi:(x,t)\mapsto \ell_\alpha(x-t)$. Let $(X_n)_{n\geq 1}$ be a sequence of i.i.d. r.v.'s with positive density. For $n\geq 1$, let $\hat q_n$ an $M$-estimator associated to $\phi$.
    \begin{enumerate}
      \item Show that $\hat q_n$ is an $\alpha$-quantile of the sample $X_1,\ldots,X_n$. To simplify matters, $\hat q_n$ will be chosen to be maximal.
      \item Find $k$ such that $\hat q_n = X_{(k)}$ where $X_{(1)}\leq\ldots\leq X_{(n)}$ are the order statistics. Show that the inequalities are strict almost surely.
      \item We want to prove that $\hat q_n$ is asymptotically normal.
      \begin{enumerate}
        \item Show that $X_1$ has a unique $\alpha$-quantile, say $q$.
        \item For $t\in \mathbb R$, show that $P(\sqrt{n}\left(\hat{q}_{n}-q\right) \leq t)=P(N \geq n \alpha)$ where $N\sim \mathcal B(n, F(q+t/\sqrt n))$ where $F$ is the cdf of $X_1$.
        \item What is the limiting distribution of $\frac{1}{\sqrt{n}}(N-n F(q+t / \sqrt{n}))$ as $n\to \infty$ ?
        \item Use Slutsky's theorem to conclude.
      \end{enumerate}
    \end{enumerate}
    }%
}

\begin{enumerate}
  \item Let $x_1,\ldots,x_n$ be fixed real numbers and $g:t \mapsto \frac 1n \sum_{i=1}^n \ell_\alpha(x_i-t)$.\newline By definition, $\hat q_n\in \argmin_{t} g(t)$. Each $t\mapsto \ell_\alpha(x_i-t)$ is a convex function, so $g$ is convex and $t$ is minimal if and only if $0\in \partial g(t)$.\newline
  Let $p(t) = |\{i\in \llbracket 1,n\rrbracket, \; x_i<t\}|$ and $q(t) = |\{i\in \llbracket 1,n\rrbracket, \; x_i>t\}|$. Subgradient calculus yields 
  $$
  \begin{aligned}
  \partial g(t) &= -\frac 1n \sum_{i=1}^n 
  \begin{cases} 
  \{\alpha-1\} &\text{if } x_i<t \\
  [\alpha-1,\alpha] &\text{if } x_i=t\\
  \{\alpha\} &\text{if } x_i>t
  \end{cases} \quad \text{where the summation is over sets}\\
  &= \{p(t)(\alpha-1) + q(t)\alpha\} + [(n-p(t)-q(t))(\alpha-1),(n-p(t)-q(t))\alpha]
  \end{aligned}$$
  Thus $\begin{aligned}[t]0\in \partial g(t)&\iff -p(t)(\alpha-1) - q(t)\alpha \in [(n-p(t)-q(t))(\alpha-1),(n-p(t)-q(t))\alpha] \\
  &\iff 0\leq n\alpha - p(t) \leq n-(p(t)+q(t)) \\
  &\iff \frac{n-p(t)}{n} \geq 1-\alpha \qquad \text{and} \qquad \frac{n-q(t)}{n} \geq \alpha
  \end{aligned}$\\
  Given the definition of $p(t)$ and $q(t)$, this can be rephrased as: $t$ is minimal if and only if it is an $\alpha$-quantile of $x_1,\ldots,x_n$.

  \item Let $x_1,\ldots,x_n$ be fixed real numbers and $p(t), q(t)$ be defined as above. We want to find the greatest $t$ such that $p(t)\leq n\alpha$ and $q(t)\leq n(1-\alpha)$ both hold. Let us show that $t^*=x_{\lfloor n\alpha \rfloor+1}$ fits the bill. By definition, $p(t^*)\leq\lfloor n\alpha \rfloor\leq n\alpha$ and $$q(t^*)\leq n-(\lfloor n\alpha\rfloor+2)+1 = n(1-\alpha)+\{n\alpha\}-1\leq n(1-\alpha)$$ 
  If $t>x_{\lfloor n\alpha \rfloor+1}$, then $p(t)\geq \lfloor n\alpha \rfloor+1 > n\alpha$, hence $t^*$ is the maximal $t$ such that $p(t)\leq n\alpha$ and $q(t)\leq n(1-\alpha)$. Thus $\hat q_n = x_{\lfloor n\alpha \rfloor+1}$.\\
  \\
  \underline{Remark}: $x_{\lceil n\alpha \rceil}$ is another $\alpha$-quantile, but it is not maximal (consider $n=6$ and $\alpha=\frac 12$). To check that it is a quantile note that $p(x_{\lceil n\alpha \rceil})\leq \lceil n\alpha \rceil -1 < n\alpha$ and $$q(x_{\lceil n\alpha \rceil})\leq n-(\lceil n\alpha \rceil+1)+1 = n-\lceil n\alpha \rceil \leq n(1-\alpha)$$ 
  \\ 
  Let $1\leq i\neq j \leq n$ and $f$ denote the density of $X_i$. \\Note that $\begin{aligned}[t]
    P(X_i=X_j) &= E(1_{X_i=X_j}) = \int 1_{x=y} dP_{(X_i,X_j)}(x,y)\\
    &=\int 1_{x=y} dP_{X_i}\otimes  dP_{X_j}(x,y) \quad \text{by independence}\\
    &=\int \int 1_{x=y}f(x)f(y)dx dy\\
    &=\int \left(\int 1_{x=y}f(x)^2dx\right) dy \quad \text{by Fubini}
  \end{aligned}$\\
  For fixed $y$, the function $x\mapsto 1_{x=y}f(x)^2$ is $0$ almost everywhere, thus $\int 1_{x=y}f(x)^2dx = 0$, hence $P(X_i=X_j) = \int 0 dy = 0$.

  \item
  \begin{enumerate}
    \item $q$ is an $\alpha$-quantile of $X_1$ if and only if $P(X_1\leq q)\geq \alpha$ and $P(X_1\geq q)\geq 1-\alpha$. Since $X_1$ has a density, its cdf $F$ is continuous. Since the density is $>0$ everywhere, $F$ is also strictly increasing, so $F$ is a continuous increasing bijection from $\mathbb R$ to $(0,1)$.\\
    \\
    Consequently there exists $q\in \mathbb R$ such that $F(q)=\alpha$, hence $P(X_1\leq q)=\alpha$ and since $X_1$ is atomless, $P(X_1\geq q) = P(X_1> q)=1 -\alpha$. Hence $q$ is an $\alpha$-quantile of $X_1$.\\\\
    If $q$ is an $\alpha$-quantile of $X_1$, we have both $P(X_1\leq q)\geq \alpha$ and $P(X_1< q)\leq \alpha$. Since $X_1$ is atomless $P(X_1\leq q)=P(X_1<q)\leq \alpha$, hence $P(X_1\leq q)= \alpha$ and $q$ is unique by the injectivity of $F$.
    \\
    \item \textbf{In this question it is essential that} $\hat q_n = X_{\lceil n\alpha \rceil}$, contrary to what's stated in Question 1.\\
    For $i\in \llbracket 1,n \rrbracket$, let $\displaystyle Y_i = 1_{X_{i} \leq \frac t{\sqrt n} +q}$ and note that $$
    \begin{aligned}
    P(\sqrt{n}\left(\hat{q}_{n}-q\right) \leq t)  
    &= P(X_{\lceil n\alpha \rceil} \leq \frac t{\sqrt n} +q)\\
    &= P(\sum_{i=1}^n Y_i \geq \lceil n\alpha \rceil)\\
    &= P(\sum_{i=1}^n Y_i \geq n\alpha)
    \end{aligned}$$
    $\sum_{i=1}^n Y_i$ has distribution $\mathcal B(n, F(t/\sqrt n + q))$ as a sum of $n$ i.i.d. Bernoulli r.v.'s.\\\\
    If $\hat q_n = X_{\lfloor n\alpha \rfloor+1}$, one gets $P(\sqrt{n}\left(\hat{q}_{n}-q\right) \leq t) = P(\sum_{i=1}^n Y_i \geq \lfloor n\alpha \rfloor+1)$ but the last term isn't necessarily equal to $P(\sum_{i=1}^n Y_i \geq n\alpha)$ (if $n\alpha\in \mathbb N$ and $m\in \mathbb N$, $m\geq n\alpha$ does not imply $m\geq \lfloor n\alpha \rfloor+1$)
    \\
    \item Note that 
    $$\begin{aligned}
      E&\left[\exp\left(it\frac{1}{\sqrt{n}}\left(N-n F\left(q+\frac t{\sqrt{n}}\right)\right)\right)\right]\\ 
      &= E\left[\exp\left(\frac{itN}{\sqrt n} \right) \right] \exp\left(-it\sqrt n F\left(q+\frac t{\sqrt{n}}\right) \right)\\
      &= \left[1+F\left(q+\frac t{\sqrt{n}}\right)\left(\exp\left(\frac{it}{\sqrt n}\right)-1 \right) \right]^n \exp\left(-it\sqrt n F\left(q+\frac t{\sqrt{n}}\right) \right)
    \end{aligned}$$
    Since the density of $X_1$ is continuous, $F$ is differentiable everywhere with $F'=f$. This provides the following asymptotic expansion for $F$:
    $$\begin{aligned}
      F\left(q+\frac t{\sqrt{n}}\right) &= F(q) + \frac t{\sqrt{n}} f(q) + o\left(\frac 1{\sqrt{n}} \right)\\
      &= \alpha + \frac t{\sqrt{n}} f(q) + o\left(\frac 1{\sqrt{n}} \right)
    \end{aligned}$$
    Let $\Log$ denote the principal branch of the logarithm. For $|z|<1$, $$\Log(1+z) = \sum_{n=1}^\infty \frac{(-1)^{n+1}}{n}z^n$$
    thus $\displaystyle \frac{|\Log(1+z)-z+\frac{z^2}2|}{|z^2|} = |z|\left|\sum_{n=3}^\infty \frac{(-1)^{n+1}}{n}z^{n-3}\right|$. $\displaystyle z\mapsto \sum_{n=3}^\infty \frac{(-1)^{n+1}}{n}z^{n-3}$ is a power series with radius $\geq 1$, it is therefore bounded over $\overline B(0,\frac 12)$. As a result $$\lim_{z\to 0} \frac{\Log(1+z)-z+\frac{z^2}2}{z^2} = 0$$
    and $\Log(1+z) = z-\frac{z^2}2 + o(z^2)$. A bit of algebra shows that 
    $$\begin{aligned}
    \Log \left[1+F\left(q+\frac t{\sqrt{n}}\right)\left(\exp\left(\frac{it}{\sqrt n}\right)-1 \right) \right] &= \Log[1 + \frac{i\alpha t}{\sqrt n} + \frac{t^2}n \left(if(q)-\frac \alpha 2\right) + o\left(\frac 1n \right)]\\
    &=\frac{i\alpha t}{\sqrt n} + \frac{t^2}n \left(if(q)-\frac \alpha 2\right) +\frac{\alpha^2 t^2}{2n} + o\left(\frac 1n \right)\\
    &=\frac{i\alpha t}{\sqrt n} + \frac{t^2}n \left(if(q)-\frac \alpha 2+\frac {\alpha^2} 2\right) + o\left(\frac 1n \right)
    \end{aligned}
    $$
    The original expectation turns into
    $$\begin{aligned}
    E\left[\exp\left(it\frac{1}{\sqrt{n}}\left(N-n F\left(q+\frac t{\sqrt{n}}\right)\right)\right)\right] &= \begin{aligned}[t]
    &\exp\left[n\left(\frac{i\alpha t}{\sqrt n} + \frac{t^2}n \left(if(q)-\frac \alpha 2+\frac {\alpha^2} 2\right) + o\left(\frac 1n \right) \right)\right]\\
    &\cdot \exp\left(-it\alpha \sqrt n -it^2 f(q) +o(1)\right)
    \end{aligned}\\
    &= \exp\left[-\frac{\alpha(1-\alpha)}2t^2 + o(1) \right]\\
    &\xrightarrow[n\to \infty]{} \exp\left[-\frac{\alpha(1-\alpha)}2t^2 \right]
    \end{aligned}
    $$
    The characteristic function of $\frac{1}{\sqrt{n}}\left(N-n F\left(q+\frac t{\sqrt{n}}\right)\right)$ converges pointwise to that of a $\mathcal N(0,\alpha(1-\alpha))$, hence $\frac{1}{\sqrt{n}}\left(N-n F\left(q+\frac t{\sqrt{n}}\right)\right)$ converges in distribution to $\mathcal N(0,\alpha(1-\alpha))$.
    \\
    \item Let $Z_n = \frac{1}{\sqrt{n}}\left(N-n F\left(q+\frac t{\sqrt{n}}\right)\right)$. Note that 
    $$P(N\geq n\alpha) = P\left(Z_n\geq \sqrt n \left(\alpha - F\left(q+\frac t{\sqrt{n}}\right) \right)\right) = P\left(-Z_n\leq \sqrt n \left(F\left(q+\frac t{\sqrt{n}}\right) - F(q) \right)\right)$$
    $\displaystyle \sqrt n \left(F\left(q+\frac t{\sqrt{n}}\right) - F(q) \right)$ is a deterministic sequence that converges (everywhere, hence almost surely, thus in probability) to $tf(q)$. We have 
    $$P(N\geq n\alpha) = P\bigg(-Z_n \underbrace{\frac{tf(q)}{\sqrt n \left(F\left(q+\frac t{\sqrt{n}}\right) - F(q) \right)} \frac{1}{f(q)}}_{\text{converges in probability to }\frac{1}{f(q)}} \leq t\bigg)$$
    By Slutsky's theorem, the random variable on the left of the $\leq$ sign converges in distribution to $-\frac{1}{f(q)} \mathcal N(0,\alpha(1-\alpha)) = \mathcal N(0,\frac{\alpha(1-\alpha)}{f(q)^2})$.\\
    Hence $P(\sqrt{n}\left(\hat{q}_{n}-q\right) \leq t) = P(N\geq n\alpha)$ converges to the cdf of a $\mathcal N(0,\frac{\alpha(1-\alpha)}{f(q)^2})$ evaluated at $t$ (and this cdf is continuous).\\
    This proves that $\sqrt{n}\left(\hat{q}_{n}-q\right)$ converges in distribution to a $\mathcal N(0,\frac{\alpha(1-\alpha)}{f(q)^2})$.
  \end{enumerate}
\end{enumerate}

\section*{Problem 4}
\noindent\fbox{%
    \parbox{\textwidth}{%
    Let $\theta>0$ and $(X_n)_{n\geq 1}$ be i.i.d. random variables following $\mathcal U([0,\theta])$. Show that the MLE $\hat\theta_n$ of $\theta$ is asymptotically exponential with convergence rate $\frac 1n$.
    }%
}
\\ \\
Let $x_1,\ldots,x_n$ be an $n$-sample. The likelihood of the model writes as $$L(\theta) = \prod_{i=1}^n \frac 1\theta 1_{[0,\theta]}(x_i)=\frac{1}{\theta^n} 1_{\min x_i\geq 0}1_{\max x_i\leq \theta}$$
If $\min x_i< 0$, $L=0$ and the MLE is not defined, so we may assume WLOG that $\min x_i\geq 0$.\\
$L$ is $0$ when $\theta <\max x_i$ and positive decreasing for $\theta \geq \max x_i$. Thus $L$ has a unique maximum at $\theta = \max x_i$, hence $\hat \theta_n = \max x_i$.\\
\\
Let us compute the cdf of $\hat \theta_n$. Let $F$ denote the cdf of $X_1$. $$\begin{aligned}
  P(\max X_i \leq t) &= P(\bigcap_{i=1}^n X_i\leq t) = F(t)^n\\
  &= \begin{cases}
    0 &\text{if } t<0\\
    \frac{t^n}{\theta^n} &\text{if } t\in [0,\theta)\\
    1 &\text{if } t\geq \theta
  \end{cases}
\end{aligned}$$
The cdf is continuous so the distribution of $\max X_i$ is atomless.\\
\\
Let $t\geq 0$. Since $\hat \theta_n$ is atomless, 
$$\begin{aligned}
  P(n(\theta-\hat \theta_n)\leq t) &= P(\hat \theta_n \geq \theta - \frac tn) = 1-P(\hat \theta_n\leq \theta-\frac tn)\\
  &= 1-\left(\theta-\frac tn \right)^n\frac{1}{\theta^n} = 1-\left(1-\frac t{\theta n} \right)^n\\
  &\xrightarrow[n\to \infty]{} 1-\exp(-\frac{t}{\theta})
\end{aligned}$$ 
If $t<0$ similar computations show that $P(n(\theta-\hat \theta_n)\leq t) \xrightarrow[n\to \infty]{} 0$\\
The limiting cdf is that of a $\mathcal E(\frac 1\theta)$ (and it is continuous), so $n(\theta-\hat \theta_n)$ converges in distribution to $\mathcal E(\frac 1\theta)$.


\newpage
\section*{Problem 5}
\noindent\fbox{%
    \parbox{\textwidth}{%
    Let $a\in \mathbb R$, $\lambda >0$ and $f:x\mapsto \lambda e^{-\lambda(x-a)}1_{x\geq a}$. Let $(X_n)_{n\geq 1}$ be an i.i.d sequence of r.v.'s with density $f$. For $n\geq 1$, let $(\hat a_n, \hat \lambda_n)$ the MLE of $(a,\lambda)$. Show that $\hat a_n$ is asymptotically exponential with convergence rate $\frac 1n$ and $\hat \lambda_n$ is asymptotically normal.
    }%
}
\\ \\
Let $x_1,\ldots,x_n$ be an $n$-sample. The likelihood of the model writes as $$L(a,\lambda) = \prod_{i=1}^n \lambda e^{-\lambda(x-a)}1_{x\geq a} = \lambda^n 1_{\min x_i \geq a} e^{-\lambda\sum_{i=1}^n(x_i-a)}$$
When $a>\min x_i$, $L(a,\lambda)=0$ and the likelihood is minimized. We may therefore assume that $a\leq \min x_i$. If $a=\frac 1n \sum_{i=1}^n x_i$, then $a=x_1=\ldots=x_n$ and $L(a,\lambda) = \lambda^n \xrightarrow[\lambda\to \infty]{} \infty$, so the MLE does not exist. We may thus assume additionally that the equalities $x_1=\ldots=x_n$ do not hold, so $a<\frac 1n \sum_{i=1}^n x_i$.\\\\
We have $\log L(a,\lambda) = n\log \lambda  -\lambda \sum_{i=1}^n(x_i-a)$. Studying the derivative w.r.t $\lambda$ shows that $\lambda\mapsto \log L(a,\lambda)$ reaches a unique maximum at $\displaystyle \lambda^*(a) = \frac{n}{\sum_{i=1}^n (x_i-a)}$ (which is well-defined given the previous assumption). Since $\log$ is stricty monotonic,  $\lambda\mapsto L(a,\lambda)$ also has its unique maximum at $\lambda^*(a)$. \\
Consider $(-\infty, \min x_i]\to \mathbb R, a\mapsto  L(a,\lambda(a^*)) = \frac{n^n}{\left[\sum_{i=1}^n(x_i-a)\right]^n}$. This function is increasing in $a$, so it reaches its maximum at $a = \min x_i$.\\ Thus $\hat a_n = \min x_i$ and $\hat \lambda_n = \lambda(\hat a_n) = \frac{n}{\sum_{i=1}^n (x_i -\min x_i)}$.\\
\\
The cdf of $X_1$ is given by $P(X_1\leq t) = \begin{cases}
  0 &\text{if } t<a\\
  1-e^{-\lambda(t-a)} &\text{if } t\geq a
\end{cases}$ and the cdf of $\min X_i$ by $P(\min X_i \leq t)=1-(1-P(X_1\leq t))^n$. \\
\\
Let $t\geq 0$. We have $P(n(\min X_i-a)\leq t) = 1-(1-(1-e^{-\lambda \frac tn}))^n = 1-e^{-\lambda t}$. For $t<0$ we get $P(n(\min X_i-a)\leq t)=0$ in a similar fashion. The cdf of $n(\min X_i-a)$ is that of a $\mathcal E(\lambda)$, hence $n(\min X_i-a)\sim \mathcal E(\lambda)$ (and remarkably this holds for finite $n$).\\
\\
$X_1$ is square-integrable with $E(X_1)=\frac 1 \lambda +a$ and $V(X_1)=\frac 1{\lambda^2}$. Note that 
$$\sqrt n \left(\frac 1n \sum_{i=1}^n X_i - \min X_i -\frac 1 \lambda \right) = \sqrt n \cdot \frac 1n \sum_{i=1}^n (X_i-E(X_i)) + \sqrt n \left(a - \min X_i \right)$$
Since $\sqrt n \cdot \sqrt n(\min X_i-a)$ converges in distribution, Question 4 from Problem 1 implies that $\sqrt n(\min X_i-a)=\o(1)$. The CLT yields the convergence in distribution of $\sqrt n \cdot \frac 1n \sum_{i=1}^n (X_i-E(X_i))$ to $\mathcal N(0,\frac{1}{\lambda^2})$. By Slutsky's theorem 
$\sqrt n \left(\frac 1n \sum_{i=1}^n X_i - \min X_i -\frac 1 \lambda \right)$ converges in distribution to $\mathcal N(0,\frac{1}{\lambda^2})$.
\\
The Delta Method applied with $x\mapsto \frac 1x$ yields the convergence in distribution of $$\sqrt n \left(\frac{1}{\frac 1n \sum_{i=1}^n X_i - \min X_i} - \lambda \right)$$ to $\mathcal N(0,\frac{1}{\lambda^2}\cdot \lambda^4)=\mathcal N(0,\lambda^2)$.


\newpage
\section*{Problem 6}
\noindent\fbox{%
    \parbox{\textwidth}{%
    Let $\theta\in \mathbb R$ and $(X_n)_{n\geq 1}$ a sequence of i.i.d. r.v.'s following $\mathcal N(\theta^3,1)$.
    \begin{enumerate}
      \item For $n\geq 1$ compute $\hat \theta_n$ the MLE of $\theta$.
      \item Show that $\hat \theta_n$ is consistent.
      \item For what values of $\theta$ is $\hat \theta_n$ asymptotically normal ?
      \item Depending on $\theta$ find $\alpha >0$ such that $|\hat \theta_n-\theta|=\O\left(\frac{1}{n^\alpha}\right)$
    \end{enumerate}
    }%
}

\begin{enumerate}
  \item Let $x_1,\ldots,x_n$ be an $n$-sample. The likelihood of the model writes as $$\begin{aligned}
  L(\theta) &= \prod_{i=1}^n \frac{1}{\sqrt{2\pi}} \exp(-\frac{(x_i-\theta^3)^2}2)\\
  &= \frac{1}{(2\pi)^{n/2}}\exp\left(-\frac 12 \sum_{i=1}^n (x_i-\theta^3)^2\right)
  \end{aligned}$$
  Thus $\displaystyle \log L(\theta) = -\frac n2 \log(2\pi)-\frac 12 \sum_{i=1}^n (\theta^3-x_i)^2$ which is a degree $6$ polynomial in $\theta$ with leading coefficient $-\frac n2$. It is therefore coercive and reaches a global maximum at a critical point. We have 
  $$(\log L)'(\theta) = 0 \iff 6\theta^2\sum_{i=1}^n (\theta^3-x_i) = 0\iff \theta=0 \text{ or } \theta = \left(\frac 1n \sum_{i=1}^n x_i\right)^{1/3}=\overline x^{1/3}$$
  Up to a constant we have $(\log L)(0)= -\frac 12 \sum_{i=1}^n x_i^2$ and $$(\log L)\left(\overline x^{1/3}\right)=-\frac 12 \sum_{i=1}^n (x_i-\overline x)^2 = -\frac 12 \left[\sum_{i=1}^n x_i^2 - \left(\sum_{i=1}^n x_i\right)^2\right]\geq (\log L)(0)$$
  The MLE is thus $\hat \theta_n = \overline x^{1/3}$.
  \\
  \item By the weak Law of Large Numbers $\overline X$ converges in probability to $\theta^3$. The continuous mapping theorem applied with $x\mapsto x^{1/3}$ yields convergence in probability of $\overline X^{1/3}$ to $\theta$, thus $\hat \theta_n$ is consistent.
  \\
  \item By the CLT $\sqrt n (\overline X-\theta^3)$ converges in distribution to $\mathcal N(0,1)$. If $\theta\neq 0$ the function $x\mapsto x^{1/3}$ is differentiable at $\theta$ and the Delta Method yields convergence in distribution of $\sqrt n (\overline X^{1/3}-\theta)$ to $\mathcal N(0,\frac 1{9\theta^4})$.\\
  Let $Y$ be a r.v. with distribution $\mathcal N(0,1)$. When $\theta=0$, combining the CLT with the continuous mapping theorem gives convergence in distribution of $n^{1/6} \overline X^{1/3}$ to $Y^{1/3}$, which rewrites as $\left[n^{1/2} \overline X^{1/3}\right] \frac{1}{n^{1/3}} \to Y^{1/3}$. If $n^{1/2} \overline X^{1/3}$ converged in distribution, Slutksy's theorem would imply that $\left[n^{1/2} \overline X^{1/3}\right] \frac{1}{n^{1/3}} \to 0$ in distribution, a contradiction. Consequently, when $\theta=0$, $\hat \theta_n$ is not asymptotically normal.
  \\
  \item For $\theta\neq 0$ we proved that $\sqrt n (\hat \theta_n-\theta) \xrightarrow[n\to \infty]{\mathcal L} \mathcal N(0,\frac 1{9\theta^4})$. Question 3 of Problem 1 implies that $\sqrt n (\hat \theta_n-\theta)$ is tight, hence $\hat \theta_n-\theta = \O\left(\frac{1}{\sqrt n}\right)$.\\
  \\
  For $\theta=0$, $n^{1/6} \hat \theta_n \xrightarrow[n\to \infty]{\mathcal L} Y^{1/3}$, and by the same argument $\hat \theta_n-\theta = \O\left(\frac{1}{n^{1/6}}\right)$.
\end{enumerate}


\end{document}