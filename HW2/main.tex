\documentclass[a4paper,11pt]{article}

\usepackage{listings}
\usepackage{lmodern}
\usepackage{amsmath,amssymb,amsthm,textcomp}
\usepackage[utf8]{inputenc}  
\usepackage[T1]{fontenc}  
\usepackage{ listings }
\usepackage{graphicx}
\usepackage{float}
\usepackage{enumitem}
\usepackage{stmaryrd}
\usepackage{bbm}
\graphicspath{ {images/} }

\usepackage{geometry}
\geometry{total={210mm,297mm},
textwidth=16cm,%
bindingoffset=0mm, top=20mm,bottom=20mm}

\usepackage{etoolbox}
\makeatletter
\preto{\@verbatim}{\topsep=0pt \partopsep=3pt }
\makeatother

\linespread{1}

\newcommand{\linia}{\rule{\linewidth}{0.5pt}}
\setlength{\parindent}{0pt}

% my own titles
\makeatletter
\renewcommand{\maketitle}{
\begin{center}
\vspace{2ex}
{\huge \textsc{\@title}}
\vspace{1ex}
\\
\linia\\
\@author 
\vspace{4ex}
\end{center}
}
\makeatother
%%%

% custom footers and headers
\usepackage{fancyhdr}
\pagestyle{fancy}
\lhead{}
\chead{}
\rhead{}

\cfoot{}
\rfoot{Page \thepage}
\renewcommand{\headrulewidth}{0pt}
\renewcommand{\footrulewidth}{0pt}
%
\DeclareMathOperator*{\argmax}{arg\,max}
\DeclareMathOperator*{\argmin}{arg\,min}
\let\o\relax
\DeclareMathOperator*{\o}{\mathit o_{\mathbb P}}
\let\O\relax
\DeclareMathOperator*{\O}{\mathit O_{\mathbb P}}
\DeclareMathOperator*{\Log}{Log}
\DeclareMathOperator*{\rk}{rk}
\DeclareMathOperator*{\inte}{int}
\DeclareMathOperator*{\cl}{cl}

%%%----------%%%----------%%%----------%%%----------%%%

\begin{document}

\title{HW2: Asymptotics 2}

\author{Gabriel ROMON}



\maketitle

\section*{Problem 1}

\noindent\fbox{%
    \parbox{\textwidth}{%
    Let $a_1,\ldots,a_q$ and $b_1,\ldots,b_q$ be vectors in $\mathbb R^d$ ($d,q\geq 1$). Suppose that $\sum_{i=1}^q b_i b_i^T$ is invertible. We want to prove that $$\sum_{i=1}^q a_i b_i^T \left(\sum_{i=1}^q b_i b_i^T\right)^{-1} \sum_{i=1}^q b_i a_i^T \preceq \sum_{i=1}^q a_i a_i^T$$
    \begin{enumerate}
      \item Show that $q\geq d$.
      \item Let $C\in \mathbb R^{q\times q}$ defined by $C_{ij} = b_i^T \left(\sum_{i=1}^q b_i b_i^T\right)^{-1} b_j$. Show that $C$ is a projection matrix. 
      \item Let $x\in \mathbb R^d$. Show that $x^T\left[\sum_{i=1}^q a_i b_i^T \left(\sum_{i=1}^q b_i b_i^T\right)^{-1} \sum_{i=1}^q b_i a_i^T\right]x$ rewrites as $y^TCy$ for some $y\in \mathbb R^q$. 
      \item Conclude.
      \end{enumerate}
    }%
}

\begin{enumerate}
  \item Since $\sum_{i=1}^q b_i b_i^T\in \mathbb R^{d\times d}$ is invertible, $d=\rk\left(\sum_{i=1}^q b_i b_i^T\right) \leq \sum_{i=1}^q \rk(b_i b_i^T) \leq \sum_{i=1}^q 1 = q$.

  \item Let us show that $C^2=C$. Let $B =\sum_{i=1}^q b_i b_i^T$. 
  $$\begin{aligned}(C^2)_{ij} &= \sum_{k=1}^q C_{ik} C_{kj}
  = \sum_{k=1}^q b_i^T B^{-1} b_k b_k^T B^{-1} b_j\\
  &=  b_i^T B^{-1} \left(\sum_{k=1}^q b_k b_k^T\right) B^{-1} b_j \\
  &= b_i^T B^{-1} B B^{-1} b_j\\
  &= C_{ij}
  \end{aligned}
  $$
  Hence $C$ is a projection matrix. Besides, since $C$ is symmetric, the projection is orthogonal.

  \item Note that $$\begin{aligned}
  x^T\left[\sum_{i=1}^q a_i b_i^T B^{-1} \sum_{k=1}^q b_k a_k^T\right]x &= \sum_{i=1}^q \sum_{k=1}^q x^Ta_i C_{ik} a_k^Tx\\
  &= \sum_{i=1}^q \sum_{k=1}^q y_i C_{ik} y_k\\
  &= y^T C y \quad \text{where } y=\begin{pmatrix}
    a_1^Tx \\ \vdots \\ a_q^Tx
  \end{pmatrix} 
  \end{aligned}
  $$

  \item Note that $x^T \sum_{i=1}^q a_i a_i^T x = y^T I_d y$, hence $$x^T \sum_{i=1}^q a_i a_i^T x - x^T\left[\sum_{i=1}^q a_i b_i^T B^{-1} \sum_{k=1}^q b_k a_k^T\right]x = y^T(I_d-C)y$$
  Since $C$ satisfies $C^2=C$ its eigenvalues are either $0$ or $1$. The eigenvalues of the symmetric matrix $I_d-C$ are thus also $0$ or $1$, hence $I_d-C\succeq 0$, so $y^T(I_d-C)y \geq 0$. This holds for all $x$, hence 
  $\sum_{i=1}^q a_i a_i^T \succeq \sum_{i=1}^q a_i b_i^T \left(\sum_{i=1}^q b_i b_i^T\right)^{-1} \sum_{i=1}^q b_i a_i^T$.
\end{enumerate}



\section*{Problem 2}

\noindent\fbox{%
    \parbox{\textwidth}{%
    Let $a,b$ be random vectors in $\mathbb R^d$ such that $E(\|a\|^2+\|b\|^2)<\infty$.
    \begin{enumerate}
      \item Show that $E(aa^T)$, $E(bb^T)$ and $E(ab^T)$ are well-defined. Show that the transpose of $E(ab^T)$ is $E(ba^T)$.\\
      In the rest of the problem we assume that $E(bb^T)$ is invertible and we want to show that $$E(ab^T) E(bb^T)^{-1} E(ba^T) \preceq E(aa^T)$$
      \item Show that the inequality of Problem 1 is a special case of this inequality. 
      \item Let $M\in \mathbb R^{p\times p}$ a block matrix defined by $M=\begin{pmatrix}
        A & B \\
        B^T & C
      \end{pmatrix}$ where $A\in \mathbb R^{k\times k}$, $B\in \mathbb R^{k\times l}$, $C\in \mathbb R^{l\times l}$ with $k+l=p$ and $A$ and $C$ symmetric. We assume that $C$ is invertible. The Schur complement of $C$ in $M$ is defined as $A-BC^{-1}B^T$. Show that $M\succeq 0$ if and only if $C$ and its Schur complement are $\succeq 0$.
      \item Let $M\in \mathbb R^{2d\times 2d}$ the block matrix defined by $M=\begin{pmatrix}
        E(aa^T) & E(ab^T) \\
        E(ba^T) & E(bb^T)
      \end{pmatrix}$
      \begin{enumerate}
        \item Show that $M\succeq 0$.
        \item Conclude.
      \end{enumerate}
      \end{enumerate}
    }%
}

\begin{enumerate}
  \item Since $E(a_i^2) \leq E(\sum_{i=1}^d a_i^2) = E(\|a\|_2^2) < \infty$, the coordinates of $a$ are in $L^2(\mathbb R)$. Thus for each $1\leq i,j\leq d$, $E(|a_i a_j|)<\infty$. Therefore $$E(\|aa^T\|_1) \leq E\left(\sum_{i=1}^d \sum_{j=1}^d |a_i a_j| \right) = \sum_{i=1}^d \sum_{j=1}^d E(|a_i a_j|)<\infty$$
  Similarly one proves that $E(\|bb^T\|_1)$, $E(\|ab^T\|_1)$ and $E(\|ba^T\|_1)$ are finite.\\
  Note that $(E(ab^T))_{ji} = E(a_j b_i) = E(b_ia_j) = (E(ba^T))_{ij}$, hence $E(ab^T)$ is the transpose of $E(ba^T)$.

  \item To avoid confusions, let $c_1,\ldots, c_q$ and $d_1,\ldots, d_q$ denote the vectors in the inequality of Problem 1. Let $\phi: \mathbb R^d\to \mathbb R^d, c_i \mapsto d_i$. Let $a$ be a random vector with distribution $\frac 1q \sum_{i=1}^q \delta_{c_i}$ and $b=\phi(a)$. $a$ and $b$ are bounded, and $E(ab^T) = E\left(\sum_{i=1}^q c_i d_i^T 1_{a=c_i}\right) = \frac 1q \sum_{i=1}^q  c_i d_i^T$.\\
  Hence $E(ab^T) E(bb^T)^{-1} E(ba^T) \preceq E(aa^T)$ rewrites as 
  $$\sum_{i=1}^q c_i d_i^T \left(\sum_{i=1}^q d_i d_i^T\right)^{-1} \sum_{i=1}^q d_i c_i^T \preceq \sum_{i=1}^q c_i c_i^T$$

  \item Let $f:\mathbb R^k\times \mathbb R^l\to \mathbb R, (u,v)\mapsto \begin{pmatrix}u^T & v^T \end{pmatrix} \begin{pmatrix}
        A & B \\
        B^T & C\end{pmatrix} \begin{pmatrix}u \\ v \end{pmatrix} = v^TCv + 2v^TB^Tu + u^TAu$\\
  If $M\succeq 0$, then for all $v\in \mathbb R^l$, $f(0,v)\geq 0$ i.e. $v^TCv \geq 0$, thus $C\succeq 0$.
  Hence $v\mapsto v^TCv$ is convex, thus $f$ is convex as a function of $v$.\\
  Let $u$ be fixed. Since $\nabla_v f(u,v) = 2Cv + 2B^Tu$, $v=-C^{-1}B^Tu$ is a critical point, hence $f(u,\cdot)$ is minimized at $-C^{-1}B^Tu$ with $0\leq f(u,-C^{-1}B^Tu) = u^T(A-BC^{-1}B^T)u$. This holds for all $u\in \mathbb R^k$, so the Schur complement of $C$ is $\succeq 0$.\\
  \\
  Suppose that $C\succeq 0$ and $A-BC^{-1}B^T\succeq 0$. For any $u\in \mathbb R^k$ and $v\in \mathbb R^l$, the previous reasoning shows that $f(u,v)\geq f(u,-C^{-1}B^Tu) = u^T(A-BC^{-1}B^T)u \geq 0$. Hence $M\succeq 0$.

  \item \begin{enumerate}
    \item We have 
    $$\begin{aligned}
      \begin{pmatrix}u^T & v^T \end{pmatrix} M \begin{pmatrix}u \\ v \end{pmatrix} 
      &= u^TE(aa^T)u + 2u^TE(ab^T)v + v^TE(bb^T)v \\
      &= E\left((u^Ta)^2+2(u^Ta)(b^Tv)+(b^Tv)^2 \right) \\
      &= E\left((u^Ta+b^Tv)^2 \right)\\
      &\geq 0
    \end{aligned}$$
    \item Remember that $E(bb^T)$ is invertible by assumption. By Question 3, the Schur complement of $E(bb^T)$ in $M$ is $\succeq 0$, that is $$E(aa^T)-E(ab^T) E(bb^T)^{-1} E(ba^T) \succeq 0$$
  \end{enumerate}
\end{enumerate}


\section*{Problem 3}

\noindent\fbox{%
    \parbox{\textwidth}{%
    Let $(X_n)_{n\geq 1}$ a sequence of i.i.d. random variables having distribution $\mathcal N(\mu,1)$ where $\mu \in \mathbb R$. Let $\alpha \in (0,1)$. For $n\geq 1$, let $\hat \mu_n = \begin{cases}
      \bar X_n & \text{if} |\bar X_n|>n^{-1/4}\\
      \alpha \bar X_n & \text{otherwise}
    \end{cases}$\\
    We want to show that $\hat \mu_n$ is asymptotically normal around $\mu$, with asymptotic variance strictly smaller than to the reciprocal of Fisher's information, for some values of $\mu$.

    \begin{enumerate}
      \item Compute Fisher's information $I(\mu)$.
      \item Suppose that $\mu=0$. 
      \begin{enumerate}
        \item Show that $P_0(|\bar X_n|>n^{-1/4}) \xrightarrow[n\to \infty]{} 0$
        \item Deduce that $\forall t\in \mathbb R$, $P_0(\sqrt n \hat \mu_n \leq t)-P_0(\sqrt n \bar X_n \leq \frac{t}{\alpha})\xrightarrow[n\to \infty]{} 0$.
        \item Conclude about $\hat \mu_n$ and compute its asymptotic variance.
      \end{enumerate}
      \item Suppose that $\mu>0$ (the case $\mu <0$ is similar).
      \begin{enumerate}
        \item Show that for $n$ sufficiently large, $P_{\mu}(|\bar X_n| \leq n^{-1/4})\leq \frac{1}{n(\mu -n^{-1/4})^2}$.
        \item Deduce that $\forall t\in \mathbb R$, $P_{\mu}(\sqrt n (\hat \mu_n -\mu) \leq t)-P_\mu(\sqrt n (\bar X_n-\mu) \leq \frac{t}{\alpha})\xrightarrow[n\to \infty]{} 0$.
        \item Conclude about $\hat \mu_n$ and compute its asymptotic variance.
      \end{enumerate}
      \item Conclude about Fisher's programme validity.
      \end{enumerate}
    }%
}

\begin{enumerate}
  \item The likelihood for a single sample is $L(\mu) = \frac{1}{\sqrt{2\pi}} \exp\left(-\frac 12 (x-\mu)^2 \right)$, thus $(\log L)'' =-1$ and $I(\mu)=1$.
  \item \begin{enumerate}
    \item The CLT yields $n^{1/4} n^{1/4} \bar X_n \xrightarrow[n\to \infty]{(d)} \mathcal N(0,1)$, hence $n^{1/4} \bar X_n \xrightarrow[n\to \infty]{P} 0$, hence $$P_0(n^{1/4} |\bar X_n| >1) \xrightarrow[n\to \infty]{} 0$$
    \item Let $t\in \mathbb R$. Note that 
    $$\begin{aligned}
    P_0(\sqrt n \hat \mu_n \leq t) &= P_0(\sqrt n \bar X_n \leq t \cap |\bar X_n|>n^{-1/4}) + P_0(\sqrt n \alpha \bar X_n \leq t \cap |\bar X_n|\leq n^{-1/4})\\
    &= P_0(\sqrt n \bar X_n \leq t \cap |\bar X_n|>n^{-1/4}) + P_0(\sqrt n \alpha \bar X_n \leq t) - P(\sqrt n \alpha \bar X_n \leq t \cap |\bar X_n|> n^{-1/4})
    \end{aligned}
    $$
    Hence $|P_0(\sqrt n \hat \mu_n \leq t)-P_0(\sqrt n \alpha \bar X_n \leq t)|\leq 2P_0(|\bar X_n|> n^{-1/4}) \xrightarrow[n\to \infty]{} 0$
    \item The CLT combined with the continuous mapping theorem shows that $\sqrt n \alpha \bar X_n$ converges in distribution to $\mathcal N(0,\alpha^2)$, hence so does $\sqrt n \hat \mu_n$.\newline
    $\hat \mu_n$ is thus asymptotically normal with asymptotic variance $\alpha^2$.
  \end{enumerate}

  \item \begin{enumerate}
    \item Let $n\geq \frac 1{\mu^4}$. Note that 
    $$\begin{aligned}
      P_\mu(|\bar X_n| \leq n^{-1/4}) &\leq P_\mu(\mu -\bar X_n \geq \mu - n^{-1/4})\\
      &= P_\mu(\bar X_n-\mu \geq \mu - n^{-1/4}) \qquad \bar X_n-\mu \text{ is symmetric}\\
      &\leq \frac{V_\mu(\hat X_n)}{\left(\mu - n^{-1/4}\right)^2}\\
      &=\frac 1n \frac{1}{\left(\mu - n^{-1/4}\right)^2}
    \end{aligned}$$ 
    \item Let $t\in \mathbb R$. Note that 
    $$\begin{aligned}
    P_\mu(\sqrt n (\hat \mu_n-\mu) \leq t) &= P_\mu(\sqrt n (\bar X_n-\mu) \leq t \cap |\bar X_n|>n^{-1/4}) + P_\mu(\sqrt n (\alpha \bar X_n-\mu) \leq t \cap |\bar X_n|\leq n^{-1/4})\\
    &= P_\mu(\sqrt n (\bar X_n-\mu) \leq t) - P_\mu(\sqrt n (\bar X_n-\mu) \leq t \cap |\bar X_n|\leq n^{-1/4}) + o(1)\\
    &= P_\mu(\sqrt n (\bar X_n-\mu) \leq t) + o(1)
    \end{aligned}$$
    \item By the CLT, $\sqrt n (\bar X_n-\mu)$ converges in distribution to $\mathcal N(0,1)$, hence so does \\$\sqrt n (\hat \mu_n-\mu)$. Therefore $\hat \mu_n$ is asymptotically normal with asymptotic variance $1$.
  \end{enumerate}
  \item Fisher's programme is not valid. There exists superefficient estimators: for some value of the parameter, the asymptotic variance is stricly smaller than the reciprocal of the information.
\end{enumerate}


\section*{Problem 4}

\noindent\fbox{%
    \parbox{\textwidth}{%
    Let $(X_n)_{n\geq 1}$ a sequence of i.i.d. random variables and let $F$ denote their cdf. Suppose that the median is unique and note it as $m$.\\
    Suppose that $F(x)-\frac 12 \sim L_2(x-m)^{\alpha}$ as $x\xrightarrow[>]{} m$ and $\frac 12 - F(x) \sim L_1(m-x)^{\alpha}$ as $x\xrightarrow[<]{} m$ where $\alpha \in (0,1]$ and $L_1,L_2>0$. Let $\hat m_n = X_{(\lceil \frac n2 \rceil)}$.\\
    Show that $n^{\frac 1{2\alpha}}(\hat m_n-m)$ converges in distribution. Interpret this result in terms of the rate of convergence of the empirical median to the median.
    }%
}
\hfill \\\\
\noindent Let $t\in \mathbb R$ and let us find the limit of $P(n^{\frac 1{2\alpha}}(\hat m_n-m) \leq t)$ as $n\to \infty$.\\
For $1\leq i \leq n$, let $Y_i=1_{X_i\leq m+\frac t{n^{\frac 1{2\alpha}}}}$. Note that $\sum_{i=1}^n Y_i$ follows $\mathcal B(n,F(m+\frac t{n^{\frac 1{2\alpha}}}))$ and 
$$P(n^{\frac 1{2\alpha}}(\hat m_n-m) \leq t) = P(\hat m_n \leq m+\frac t{n^{\frac 1{2\alpha}}}) 
= P\left(\sum_{i=1}^n Y_i \geq \lceil \frac n2 \rceil\right) = P\left(\sum_{i=1}^n Y_i \geq  \frac n2\right)$$
Let $N=\sum_{i=1}^n Y_i$. Computations analoguous to those carried out in Question 3.c) of Problem 3 in HW1 show that $\frac{1}{\sqrt n}\left(N-nF\left(m+\frac t{n^{\frac 1{2\alpha}}}\right) \right)$ converges in distribution to $\mathcal N(0,\frac 14)$, regardless of $L_1$ and $L_2$.\\
Let $Z_n=\frac{1}{\sqrt n}\left(N-nF\left(m+\frac t{n^{\frac 1{2\alpha}}}\right) \right)$ and note that 
$$P\left(\sum_{i=1}^n Y_i \geq  \frac n2\right) =  P\left[-Z_n\leq \sqrt n\left(F\left(m+\frac t{n^{\frac 1{2\alpha}}}\right)-\frac 12 \right)\right]$$
When $t\geq 0$, the deterministic sequence $\sqrt n\left(F\left(m+\frac t{n^{\frac 1{2\alpha}}}\right)-\frac 12 \right)$ converges to $L_2t^\alpha$ 
and $$P\left[-Z_n\leq \sqrt n\left(F\left(m+\frac t{n^{\frac 1{2\alpha}}}\right)-\frac 12 \right)\right] = P\bigg[\underbrace{-2Z_n \frac{L_2t^\alpha}{\sqrt n\left(F\left(m+\frac t{n^{\frac 1{2\alpha}}}\right)-\frac 12 \right)}}_{\xrightarrow[n\to \infty]{(d)} \mathcal N(0,1) \text{ by Slutsky}}  \leq 2L_2t^\alpha\bigg]$$
hence $P(n^{\frac 1{2\alpha}}(\hat m_n-m) \leq t) \xrightarrow[n\to \infty]{} \Phi(2L_2t^\alpha)$ where $\Phi$ is the cdf of $\mathcal N(0,1)$.
\\ \\
When $t\leq 0$ the deterministic sequence converges to $-L_1(-t)^{\alpha}$ and 
$$P\left[-Z_n\leq \sqrt n\left(F\left(m+\frac t{n^{\frac 1{2\alpha}}}\right)-\frac 12 \right)\right] = P\bigg[\underbrace{-2Z_n \frac{L_1(-t)^\alpha}{\sqrt n\left(\frac 12 - F\left(m+\frac t{n^{\frac 1{2\alpha}}}\right) \right)}}_{\xrightarrow[n\to \infty]{(d)} \mathcal N(0,1) \text{ by Slutsky}}  \leq -L_1(-t)^{\alpha} \bigg]$$
hence $P(n^{\frac 1{2\alpha}}(\hat m_n-m) \leq t) \xrightarrow[n\to \infty]{} \Phi(-2L_1(-t)^\alpha)$\\
\\
The piecewise function $\varphi: t\mapsto \begin{cases}
  \Phi(2L_2t^\alpha) &\text{if } t\geq 0\\
  \Phi(-2L_1(-t)^\alpha) &\text{if } t\leq 0
\end{cases}$ is continuous, increasing, with $\lim_\infty \varphi(x)=1$ and $\lim_{-\infty} \varphi(x)=0$. $\varphi$ is thus the cdf of some r.v. $Y$, and the cdf of $n^{\frac 1{2\alpha}}(\hat m_n-m)$ converges pointwise to $\varphi$. Since $\varphi$ is continuous, we may conclude that $$n^{\frac 1{2\alpha}}(\hat m_n-m) \xrightarrow[n\to \infty]{(d)} Y$$ where 
$F_Y: t\mapsto \begin{cases}
  \Phi(2L_2t^\alpha) &\text{if } t\geq 0\\
  \Phi(-2L_1(-t)^\alpha) &\text{if } t\leq 0
\end{cases}$



\section*{Problem 5}

\noindent\fbox{%
    \parbox{\textwidth}{%
    Let $(X_n)_{n\geq 1}$ be a sequence i.i.d. r.v's with values in some interval $I\subset \mathbb R$ having a density $f$. Suppose that $f$ is positive over $I$ and continuous. Let $\phi: (x,t)\mapsto |x-t|-|x|$, $\Phi:t\mapsto E(\phi(X_1,t))$ and $\Phi_n:t\mapsto \frac 1n \sum_{i=1}^n \phi(X_i,t)$.
    \begin{enumerate}
      \item Show that $X_1$ has a unique median, say $m$, and that $m\in \inte I$.
      \item Show that $\Phi$ is well-defined.
      \item Show that $\Phi$ is twice-differentiable at $m$ with $\Phi''(m)>0$.
      \item Show that $\Phi$ has a unique minimum attained at $m$.
      \item Deduce that an empirical median $\hat m_n$ computed from $X_1,\ldots, X_n$ is asymptotically normal and compute its asymptotic variance.
      \end{enumerate}
    }%
}

\begin{enumerate}
  \item Since $X_1$ has a density, its cdf $F$ is continuous. Note that $F$ (as a function defined on $\overline{\mathbb R}$) is $0$ over $[-\infty, \inf I]$ and $1$ over $[\sup I, \infty]$. Since $f>0$ over $I$, $F$ is strictly increasing over $\inte I$. By the intermediate value theorem, there exists $m\in \inte I$ such that $F(m)=\frac 12$ (which is necessary and sufficient for $m$ to be a median because $X_1$ is atomless). Since $F$ is strictly increasing over $\inte I$, there is no other median.
  \item By the reversed triangle inequality, $||X-t|-|X|| \leq |t|$, thus $|X-t|-|X|$ is bounded and its expectation is well-defined.
  \item Note that $E(|t-X|-|X|) = E(|-t-(-X)|-|-X|)$, so by replacing $t$ with $-t$ and $X$ with $-X$ we may suppose WLOG that $t\geq 0$ in the expressions to come.
  $$\begin{aligned}
    \Phi(t) &= E((t-X-X)1_{t-X>0}1_{X>0} + (t-X+X)1_{t-X>0}1_{X<0} + (X-t-X)1_{t-X<0})\\
    &= E(t1_{t>X>0} -2X1_{t>X>0} + t1_{X<0} -t1_{t<X}) \qquad \text{since } t\geq 0\\
    &= E(t1_{t>X} -2X1_{t>X>0} -t + t1_{t>X})\\
    &= 2tF(t) -t-2\int_0^t xf(x) dx
  \end{aligned}$$
  Note how all the inequalities are strict because $X$ is atomless. For $t\leq 0$, replacing $t$ with $-t$ and $X$ with $-X$ yields 
  $$\begin{aligned}
    \Phi(t) &= E(-t1_{-t>-X} +2X1_{-t>-X>0} +t -t1_{-t>-X}) \\
    &=t-2tP(X>t)+2\int_t^0xf(x)dx \\
    &= 2tF(t) -t-2\int_0^t xf(x) dx
  \end{aligned}$$
  Hence $\Phi(t) = 2tF(t) -t-2\int_0^t xf(x) dx$ holds for all $t\in \mathbb R$.\\
  Since $f$ is continuous, $F$ is differentiable with $F'=f$, thus $\Phi$ is differentiable and computations yield $\Phi'=2F-1$. Hence $\Phi$ is also twice differentiable with $\Phi''=2f$. Since $f$ is positive over $I$ and $m\in \inte I$, we have $\Phi''(m)>0$.
  \item We have $\Phi'(t)>0 \iff F(t)>\frac 12\iff t>m$ and $\Phi'(t)<0 \iff t<m$ hence $\Phi$ has a unique minimum and it is attained at $m$.
  \item 
  Note that up to a constant $\Phi_n(t)=\frac 1n \sum_{i=1}^n |X_i-t|$, which is minimized in $t$ by any median $\hat m_n$ of $X_1,\ldots, X_n$ (take $\alpha=\frac 12$ in Question 1 of Problem 3 in HW1).\\
  \\
  For all $x\in \mathbb R$, $t\mapsto \phi(x,t)$ is convex, $\Phi$ has a unique minimum at $m$ and $\Phi$ is twice differentiable at $m$ with $\Phi''(m)=2f(m)>0$.\\
  Let $x$ be fixed. A subgradient of $t\mapsto \phi(x,t)$ at $t$ is $\begin{cases}
    -1 &\text{if } t<x\\
    0 &\text{if } t=x\\
    1 &\text{if } t>x
  \end{cases}$, hence a subgradient of $t\mapsto \phi(X,t)$ at $t$ is $g(X,t)= 1_{t>X}-1_{t<X}$ and this subgradient is clearly measurable.\\
  Since $X$ is atomless, $g(X,t)^2=1$ almost surely. By the theorem on convex M-estimation covered in class, $\hat m_n \xrightarrow[n\to \infty]{a.s.} m$ and $\sqrt n (\hat m_n-m) \xrightarrow[n\to \infty]{(d)} \mathcal N(0,\frac{1}{4f^2(m)})$.\\
  This is consistent with the result of Question 3d) of Problem 3 in HW1 ($\alpha=\frac 12$).
\end{enumerate}

\newpage
\section*{Problem 6}

\noindent\fbox{%
    \parbox{\textwidth}{%
    Let $(X_n)_{n\geq 1}$ be a sequence i.i.d. r.v's with values in some interval $I\subset \mathbb R$ having a density $f$. Suppose that $f$ is positive over $I$ and continuous. Let $c>0$ and $$\ell_c:u\mapsto \begin{cases}
      u^2 &\text{if } |u|\leq c\\
      2c|u|-c^2 &\text{otherwise}
    \end{cases}$$
    Let $\phi:(x,t)\mapsto \ell_c(x-t)-2c|x|$, $\Phi:t\mapsto E(\phi(X_1,t))$ and $\Phi_n:t\mapsto \frac 1n \sum_{i=1}^n \phi(X_i,t)$.
    \begin{enumerate}
      \item Show that $\Phi$ is well-defined.
      \item Show that $\Phi$ has a unique minimum attained at say $m$, with $m\in \inte I$. Show that $\Phi$ is twice-differentiable at $m$ and $\Phi''(m)>0$.
      \item Show that for all $n\geq 1$, $\Phi_n$ admits a minimizer $\hat m_n$.
      \item Show that $\hat m_n$ is asymptotically normal and compute its asymptotic variance.
      \end{enumerate}
    }%
}

\begin{enumerate}
  \item Note that $$\ell_c(X-t)-2c|X| = E\left[(X-t)^21_{|X-t|\leq c} + \left(2c(|X-t|-|X|)-c^2 \right)1_{|X-t|>c} - 2c|X|1_{|X-t|\leq c}\right]$$
  Each of the three summands is bounded since: \\$(X-t)^21_{|X-t|\leq c} \leq c^2$\\
  $\big|2c(|X-t|-|X|)-c^2 \big|1_{|X-t|>c}\leq 2c|t|+c^2$\\
  $2c|X|1_{|X-t|\leq c}\leq 2c(c+|t|)$.\\
  Hence $\Phi$ is well-defined.
\end{enumerate}

\end{document}