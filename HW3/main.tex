\documentclass[a4paper,11pt]{article}

\usepackage{listings}
\usepackage{lmodern}
\usepackage{amsmath,amssymb,amsthm,textcomp}
\usepackage[utf8]{inputenc}  
\usepackage[T1]{fontenc}  
\usepackage{ listings }
\usepackage{graphicx}
\usepackage{float}
\usepackage{enumitem}
\usepackage{stmaryrd}
\usepackage{bbm}
\graphicspath{ {images/} }

\usepackage{geometry}
\geometry{total={210mm,297mm},
textwidth=16cm,%
bindingoffset=0mm, top=20mm,bottom=20mm}

\usepackage{etoolbox}
\makeatletter
\preto{\@verbatim}{\topsep=0pt \partopsep=3pt }
\makeatother

\linespread{1}

\newcommand{\linia}{\rule{\linewidth}{0.5pt}}
\setlength{\parindent}{0pt}

% my own titles
\makeatletter
\renewcommand{\maketitle}{
\begin{center}
\vspace{2ex}
{\huge \textsc{\@title}}
\vspace{1ex}
\\
\linia\\
\@author 
\vspace{4ex}
\end{center}
}
\makeatother
%%%

% custom footers and headers
\usepackage{fancyhdr}
\pagestyle{fancy}
\lhead{}
\chead{}
\rhead{}

\cfoot{}
\rfoot{Page \thepage}
\renewcommand{\headrulewidth}{0pt}
\renewcommand{\footrulewidth}{0pt}
%
\DeclareMathOperator*{\argmax}{arg\,max}
\DeclareMathOperator*{\argmin}{arg\,min}
\let\o\relax
\DeclareMathOperator*{\o}{\mathit o_{\mathbb P}}
\let\O\relax
\DeclareMathOperator*{\O}{\mathit O_{\mathbb P}}
\DeclareMathOperator*{\Log}{Log}
\DeclareMathOperator*{\rk}{rk}
\DeclareMathOperator*{\inte}{int}
\DeclareMathOperator*{\cl}{cl}

%%%----------%%%----------%%%----------%%%----------%%%

\begin{document}

\title{HW3: Asymptotics 3}

\author{Gabriel ROMON}



\maketitle

\section*{Problem 1}

\noindent\fbox{%
    \parbox{\textwidth}{%
    Let $f:\mathbb R^d \to \mathbb R$ be a convex function and $a\in \mathbb R^d$.
    \begin{enumerate}
      \item Suppose that $\|x-a\|_2=1\implies f(x)>f(a)$. Show that $f$ is bounded below and that its minimum is attained only at points such that $\|x-a\|_2<1$.
      \item More generally, let $K$ be a compact set with $a\in \inte K$. Suppose that $x\in \partial K \implies f(x)>f(a)$. Show that $f$ is bounded below and that its minimum is attained only in the interior of $K$.
      \end{enumerate}
    }%
}



\begin{enumerate}
  \item Consider $x$ such that $\|x-a\|_2>1$ and let $\phi:[0,\infty)\to \mathbb R^d, t \mapsto f(a+t\frac{x-a}{\|x-a\|_2})$.\\ By definition, $\phi$ is convex, $\phi(0)=f(a)$, $\phi(1)=f(a+\frac{x-a}{\|x-a\|_2}) > f(a)$ and $\phi(\|x-a\|_2)=f(x)$. By the inequality on slopes of convex real functions, $$\frac{\phi(\|x-a\|_2)-\phi(0)}{\|x-a\|_2} \geq \frac{\phi(1)-\phi(0)}{1}$$ thus $\frac{f(x)-f(a)}{\|x-a\|_2} \geq f(a+\frac{x-a}{\|x-a\|_2}) -f(a)$, hence $\begin{aligned}[t]f(x)-f(a) &\geq \|x-a\|_2 \left(f(a+\frac{x-a}{\|x-a\|_2}) -f(a)\right)\\
  &>0 \end{aligned}$
  Hence $\|x-a\|_2>1 \implies f(x)>f(a)$. \\
  $f$ is convex on the open set $B(a,\frac 32)$, hence continuous over $B(a,\frac 32)$ and thus continuous over $\overline B(a,1)$. It reaches therefore a minimum over $\overline B(a,1)$, which is also a global minimum since $\|x-a\|_2>1 \implies f(x)>f(a)$. Besides, since $\|x-a\|_2=1\implies f(x)>f(a)$, the minimum is attained only at points such that $\|x-a\|_2<1$.

  \item Consider $x\notin K$ and let $A=\{t\geq 0, \; a+t\frac{x-a}{\|x-a\|_2} \in K\}$. $0\in A$ so $A$ is non-empty, and since $K$ is bounded, $A$ is bounded above. Let $t_0=\sup A$. Let us show that $a+t_0\frac{x-a}{\|x-a\|_2}\in \partial K$. Since $K$ is closed we have clearly $a+t_0\frac{x-a}{\|x-a\|_2}\in K$. Let $\varepsilon >0$. Since $t_0+\varepsilon \notin A$ we have $a+(t_0+\varepsilon) \frac{x-a}{\|x-a\|_2} \notin K$ and since $\lvert\lvert a+(t_0+\varepsilon)\frac{x-a}{\|x-a\|_2} - \left(a+t_0\frac{x-a}{\|x-a\|_2} \right)\rvert\rvert_2  = \varepsilon$, we have $a+t_0\frac{x-a}{\|x-a\|_2}\notin \inte K$, thus $a+t_0\frac{x-a}{\|x-a\|_2}\in \partial K$.\\\\
  Since $\|x-a\|_2\notin A$, we have $\|x-a\|_2 > t_0$ and the same argument as in 1. shows that  $$f(x)-f(a) \geq \frac{\|x-a\|_2}{t_0} \left(f(a+t_0\frac{x-a}{\|x-a\|_2}) -f(a)\right)>0$$
  $K$ being compact, it is bounded so there exists an open ball $B$ such that $K\subset B$. $f$ is convex over $B$, hence continuous over $K$, so $f$ reaches a minimum over $K$. Since $x\notin K\implies f(x)>f(a)$, this minimum is global, and since $x\in \partial K\implies f(x)>f(a)$, it is attained only at points in the interior of $K$.

\end{enumerate}

\end{document}